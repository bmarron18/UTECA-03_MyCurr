%%%%%%%%%%%%%%%%%%%%%%%%%%%%%%%%%%%%%%%%%
% Quality Report Format 
% Derived from The Legrand Orange Book LaTeX Template (Overleaf)
%%%%%%%%%%%%%%%%%%%%%%%%%%%%%%%%%%%%%%%%%



% When you first open the template, compile it from the command line with the 
% commands below to make sure your LaTeX distribution is configured correctly:
% Re-run to get TOC
% 1) pdflatex Retaining_Volunteers_20171103
% 2) biber Retaining_Volunteers_20171103
% 3) pdflatex Retaining_Volunteers_20171103 x 2

% run LaTex -> PDF (Latexmk) in LaTexilla to get .idx file
% then
% 1) pdflatex Retaining_Volunteers_20171103
% 2) makeindex Retaining_Volunteers_20171103.idx -s StyleInd.ist
% 3) biber Retaining_Volunteers_20171103
% 4) pdflatex Retaining_Volunteers_20171103 x 2
% 

%%%%%%%%%%%%%%%%%%%%%%%%%%%%%%%%
%GRAPHICS
%%%%%%%%%%%%%%%%%%%%%%%%%%%%%%%%%
% graphics for cover photo and chapter photos must be .jpeg



%----------------------------------------------------------------------------------------
%	PACKAGES AND OTHER DOCUMENT CONFIGURATIONS
%----------------------------------------------------------------------------------------

\documentclass[11pt,fleqn]{book} % Default font size and left-justified equations

\input{structure} % Insert the commands.tex file which contains the majority of the structure behind the template

\usepackage[top=3cm,bottom=3cm,left=3.2cm,right=3.2cm,headsep=10pt,letterpaper]{geometry} % Page margins
\usepackage{lipsum}
\usepackage{xcolor} % Required for specifying colors by name
\definecolor{ocre}{RGB}{51,102,0} 
\definecolor{lightgray}{RGB}{229,229,229}
\usepackage{csquotes}


% Font Settings
\usepackage{avant} % Use the Avantgarde font for headings
%\usepackage{times} % Use the Times font for headings
\usepackage{mathptmx} % Use the Adobe Times Roman as the default text font together with math symbols from the Sym­bol, Chancery and Com­puter Modern fonts

\usepackage{microtype} % Slightly tweak font spacing for aesthetics
\usepackage[utf8]{inputenc} % Required for including letters with accents
\usepackage[T1]{fontenc} % Use 8-bit encoding that has 256 glyphs


% MATHS PACKAGE
\usepackage{amsmath}
\usepackage{calc}

% GRAPHICS
\usepackage{tikz}
\usetikzlibrary{matrix}
\newcommand*{\horzbar}{\rule[0.05ex]{2.5ex}{0.5pt}}


% VERBATIM PACKAGE
\usepackage{verbatim}



% BIBLIOGRAPHY
\usepackage[american]{babel} 
\usepackage[style=apa,
            sorting=nyt,
            sortcites=true,
            autopunct=true,
            hyperref=true,
            maxcitenames=2,
            mincitenames=1,
            maxbibnames=10,
            backref=true,
            doi=false,
            url=false,
            backend=biber]{biblatex}
\addbibresource{/home/bruce/Desktop/BibLatex/My_Library_20170125.bib} % BibTeX bibliography file
%\defbibheading{bibempty}{}
\DeclareLanguageMapping{american}{american-apa}    % avoid yearmonthday problems


% INDEX
\usepackage{imakeidx}
\makeindex



% SPECIALTY COMMANDS
\newcommand\ddfrac[2]{\frac{\displaystyle #1}{\displaystyle #2}}

% $\ddfrac{TP+TN}{\Sigma Total Population}$


%%%%%%%%%%%%%%%%%%%%%%%%%%%%%%%%%%%%
%    BEGIN DOCUMENT
%%%%%%%%%%%%%%%%%%%%%%%%%%%%%%%%%%%%

% HELPFUL TIPS
% \autocite{ABC01}      %for et al.
% \parencite{}
% \textcite{}            % Shields (2000) ...

\begin{document}

\let\cleardoublepage\clearpage

%----------------------------------------------------------------------------------------
%	TITLE PAGE
%----------------------------------------------------------------------------------------

\begingroup
\thispagestyle{empty}
\AddToShipoutPicture*{\put(0,0){\includegraphics[scale=0.95]{graphics/PortlandNETs2.jpeg}}} % Image background; * means only on this page
\vspace*{6cm}     %from the top of page
\centering{
\normalfont\fontsize{35}{35}\sffamily\selectfont
\textbf{ Retaining Volunteers: \\
Creative Options and Recommendations}\\  % Book title
\vspace*{7cm}    % between title and client/author lines
{\LARGE Produced for: \\
Portland Bureau of Emergency Management\\
\vspace*{1cm}   % between client and author
Produced by: \\
Bruce D. Marron \\}  % Author name
} 
\endgroup

%----------------------------------------------------------------------------------------
%	COPYRIGHT PAGE
%----------------------------------------------------------------------------------------

\newpage
~\vfill
\thispagestyle{empty}

\noindent Cover image, \copyright\ Portland NET, 2017\\ % Copyright notice
\noindent Chapter image, \copyright\ Los Angeles County Community Disaster Resilience, 2017\\ % Copyright notice
\noindent \textit{Published 15 November 2017} % Printing/edition date

%\noindent \textsc{Projet Janvier-Mars 2015, Université de Bourgogne}\\
%\noindent Ce projet a été encadré par Hervé CARDOT.\\ % License information



%----------------------------------------------------------------------------------------
%	TABLE OF CONTENTS
%----------------------------------------------------------------------------------------

\chapterimage{graphics/LAResilience.jpeg} % heading image

\pagestyle{empty} % No headers

\renewcommand\contentsname{Table of Contents}
\renewcommand{\bibname}{Bibliography}
\tableofcontents% Print the table of contents itself

%\cleardoublepage % Forces the first chapter to start on an odd page so it's on the right

\pagestyle{fancy} % Print headers again

%%%%%%%%%%%%%%%%%%%%%%%%%%%%%%%%%%%%%%%
%	CHAPTER 1
\chapterimage{graphics/LAResilience.jpeg} % Chapter heading image
\chapter{A Brief Historical Perspective}
%%%%%%%%%%%%%%%%%%%%%%%%%%%%%%%%%%%%%%%%%

The inauguration of present day Community Emergency Response Teams (CERTs) in the US typically dates to 1985 when, spurred by the Los Angeles Fire Department (LAFD), the city of Los Angeles sent representatives to Tokyo, Japan to observe a community-wide earthquake response drill \autocite{simpson_community_2001}. The LAFD concerns related to community emergency response were most likely generated by the September 19, 1985 earthquake in Mexico City in which hundreds of volunteers, lacking adequate training, perished along with thousands of others. \textcite{simpson_community_2001} points to an earlier beginning and suggests that CERT programs could be considered the progeny of the Civil Defense program, the Cold War era effort aimed at preparing citizens for disaster spawned by nuclear war. Certainly there are similarities in the two programs, but it is their differences which can provide insights into maintaining the viability of present day CERT programs. The first notable difference is in perceived threat: the Civil Defense program faded as the public's level of perceived threat of nuclear war diminished. In contrast, CERT programs continue to flourish as more real disaster challenges are recognized by more and more communities. Florida must take the credit for having the insight to move the Los Angeles earthquake-centered model to address local risk, in this case hurricanes. More important is the second difference. \textcite{simpson_community_2001} believes that Civil Defense programs lost traction because \enquote{they did not carry out neighborhood building activities}. The Civil Defense program ultimately failed to be inclusive and relevant.\\

\noindent There are two additional historically relevant events that have directly affected the development of CERT programs. The first is federal involvement. Beginning in 1993 the Federal Emergency Management Agency (FEMA) began offering \enquote{train-the-trainer} courses at the Emergency Management Institute (EMI) in Emmitsburg, Maryland. Since then FEMA has changed the nature of the CERT program by encouraging \enquote{disaster resilient communities} and providing institutionalization, validation, standardization, facilitation, and promotion \autocite{simpson_community_2001}. Early in the development of CERT programs, communities struggled not only for identity in terms of name and mission, but also for access to training guidelines and materials. Now, with some degree of federal sponsorship, there are readily available materials and local CERT programs can feel confident and legitimate in their mission and training. The second historically relevant event was Hurricane Andrew. In 1992 Hurricane Andrew struck Florida and it became clear that the assumption of a 72-hr official response time was wrong. CERT programs, including the Portland NET program, now prepare volunteers and communities to be self-reliant for at least two weeks. Community members that have internalized the reality of official response time following a major disaster are likely to recognize the real value in becoming a CERT volunteer.


%%%%%%%%%%%%%%%%%%%%%%%%%%%%%%%%%%%%%%%%%%%%%%%
\chapterimage{graphics/LAResilience.jpeg} % Chapter heading image
\chapter{A Brief Look at the Literature}
%%%%%%%%%%%%%%%%%%%%%%%%%%%%%%%%%%%%%%%%%%%%%%%

\section{Volunteerism}
\vspace{1em}

Developing effective strategies for the recruitment and retention of volunteers requires a fundamental understanding of volunteer motivations and expectations. Understanding the motivations that create volunteers is the key to recruitment efforts; understanding the expectations that when fulfilled, keep volunteers coming back is the key to retention efforts. Volunteer motivation is a complex state space along the axes of humanitarianism and egoism. That is to say that people volunteer for many reasons such as career development, personal enhancement, social responsibility, concern for society, and social enjoyment \autocite{shields_young_2009}. Ultimately, volunteers are motivated by both self-interest and altruism, often acting to reinforce each other. For example, \textcite{fahey_training_2002} notes that the Volunteer Ambulance Corps of Tasmania program attracts people who are interested in \enquote*{learning new skills} as well as those who have the urge to \enquote*{assist the community}. There is some evidence that younger volunteers may be motivated more by financial and career success while older volunteers may be motivated by social responsibility \autocite{shields_young_2009}.



\section{Recruitment and Retention of Volunteers}
\vspace{1em}

Recruitment efforts begin with organization evaluation. A program manager responsible for volunteer recruitment would be expected (1) to evaluate the organization's volunteer climate, (2) to detail the types of volunteer jobs that are needed and develop appropriate volunteer job descriptions, (3) to accurately  describe and characterize the organization's current volunteers, and (4) to understand the influences that affect people's decision to volunteer. Once this background work is completed a volunteer program manager may consider the development of different promotional appeals. For example, some authors suggest that recruitment efforts should target different age subgroups, presumably because these subgroups have different motivations. \textcite{shields_young_2009}, however, emphasizes that \enquote{maximum marketing efficiency} is to be gained by concurrently highlighting a broad spectrum of themes. These are the themes that relate directly to the primary reservoirs of volunteer inspiration: humanitarianism, social responsibility, and personal development.

Regardless of style, the central message in any recruitment appeal must be the importance of the organization's mission \autocite{wymer_jr._conceptual_2001}. Consistently, volunteers report that the main reasons for joining an organization (wanting to give to the
community, maintaining and expanding skills) must be deductively apparent in the organization's mission statement \autocite{ranse_engaging_2010}. Thus, a recruitment message should always have an explicit statement of mission and logical action links derived from the mission statement that support the main reasons for volunteering. \textcite{shields_young_2009} offers a simple but effective message scheme, which when adapted for the Portland NET program, would look something like,\\
\fbox{
\begin{minipage} [t][][c]{0.80\linewidth}
\includegraphics[scale=0.10]{graphics/PBEM_logo.jpg}
\textbf{Portland Bureau of Emergency Management}\\
\textit{Disaster risk reduction through leadership and coordination.}\\

Following a major disaster, first responders who provide fire and medical services will not be able to meet the demand for these services. Factors such as number of victims, communication failures and road blockages will prevent people from accessing emergency services that they have come to expect at a moment's notice through 911. People will have to rely on each other for help in order to meet their immediate life saving and life sustaining needs. \\

\includegraphics[scale=0.25]{graphics/NET_logo.jpg}
\textbf{Portland Neighborhood Emergency Team Program}\\
 When you volunteer for Portland NET ...\\
...you will actively participate in important activities that make a difference.\\
...you will have the opportunity to better yourself personally and professionally.\\
...you will make a difference in the lives of others.\\
...you will make connections with others.
 \end{minipage}
}

\noindent Volunteers remain with an organization when they experience high levels of connectedness, uniqueness, engagement, appreciation and empowerment \autocite{shields_young_2009}. High levels of connectedness result from being part of a group with which one shares goals, values, respect, and trust. Volunteers experience a feeling of uniqueness when they believe that they contribute a unique combination of talents and personality to the organization. Volunteers experience engagement when they have opportunities to demonstrate of their skills, and they experience appreciation and personal empowerment when they recognized as having made a real difference. \\



\vspace{2em}
\section{Some Notable Programs}
\vspace{1em}

\autocite{simpson_community_2001}
** The other activity that has kept the Orlando CERT visible is the
sponsoring of an annual drill for CERT graduates. The
drill is held at Universal Studios, with more than 5,000
spent on pyrotechnics alone (Simpson 2000a). Moulage
and victim makeup are donated by Universal Studios, and
the theme park is used after-hours as the location for rec-
reating a disaster—replete with overturned buses,
wrecked airplanes, and multiple casualties.

**As an example of regional efforts, in April 1996
BayNET assisted in the holding of a region-wide Earth-
quake Drill Day, in which all community programs were
encouraged to participate.

** Kansas City NET Rodeo





\vspace{2em}
\section{The Concept of Resilience}
\vspace{1em}




%\vspace{1em}

%%%%%%%%%%%%%%%%%%%%%%%%%%%%%%%%%%%%%%%%%%%%%%%%%
\chapterimage{graphics/LAResilience.jpeg} % Chapter heading image
\chapter{Recommendations}
%%%%%%%%%%%%%%%%%%%%%%%%%%%%%%%%%%%%%%%%%%%%%%%%%%%%

\section{Recommendations}

*** Bicycle powered corps
*** outreach ambassador ==> be friend (event attendance)
*** cultivate some appropriate climate of fun \autocite{karl_give_2008}
** longitudinal study Cross-sectional studies make comparisons at a single point in time, whereas longitudinal studies make comparisons over time. researchers might start with a cross-sectional study to first establish whether there are links or associations between certain variables. Then they would set up a longitudinal study to study cause and effect.\\
** phone app like iNaturalist for citizen science
1) risk assessment (citizen science for risk assessment)
Chapter 4: Emergency Management Program Elements
An Accredited Emergency Management Program has a Hazard Identification, Risk Assessment (HIRA)
and Consequence Analysis.
identifies the natural and human-caused hazards
that potentially impact the jurisdiction using multiple sources. The Emergency Management
Program assesses the risk and vulnerability of people, property, the environment, and its own
operations from these hazards.
2) help with developing Commodity Point of Distribution plan
3) help with developing a plan to coordinate with the private sector during emergency response and recovery
4) a model to be used by the City’s Damage Assessment Plan (a plan that addresses both public and private sector damage, and allows
residents to self-report losses)







** individualized training programs noting that VAO members rated training highly\\
** training is not mentioned in this context.
However, we found training should be considered not
only as a recruitment tool, but also as a strong retention
tool. Appropriate training should be high quality, flexible,
timely, and meet set standards. Poorly delivered or
constantly re-scheduled training is a disincentive to
VAO and the same principles are likely to apply for
other emergency service volunteers. All emergency
services aiming to recruit and retain volunteers should
investigate the flexibility, quality and timeliness of
the training they deliver to ensure it reinforces the
motivations of their workforce\\
** the focus of recruitment should be retention\\
** volunteers require attention on their anniversary and after completing big jobs\\
** cultivate their role identity (increases self-esteem and thus retention\\

** see Wymer and Starnes list of acknowledgements (p.17)\\
** tenure for team leaders there may be possible negative effects of volunteers
staying too long. It’s possible that long-term volunteers may become too powerful or too complacent therefore, hindering rather than contributing to organizational effectiveness.\\

** Ellis (1994) recommends that VPMs should recruit for specific volunteer
jobs, to fill specific volunteer roles. This contrasts to a general “call to volun-
teer” appeal that an organization may generate using mass media, such as a
public service announcement or a paid advertisement \autocite{wymer_jr._conceptual_2001}\\



%%%%% LITERATURE %%%%%%%%%%%%%%%%%%%%%%%%%%
\autocite{ranse_engaging_2010}
developed and delivered training is seen as a retention
tool for some emergency management organisations
(Fahey, et al., 2003) and poorly delivered and inflexible
training will result in the loss of volunteers (Fahey, et al,
2003). In addition to training in core business skills and
knowledge, training should include extended skills and
attributes such as leadership and coordination skills
(Aitken, 2000). However, it is important that effective
training is not considered the only tool for retaining
volunteers (Fahey, et al, 2002). Equally important is
promoting a community focus

This research demonstrated similar reasons for joining
a volunteer emergency management organisation as
outlined in the literature, such as, wanting to give to the
community, maintaining and expanding skills


\autocite{fahey_training_2002}\\
 The Volunteer Ambulance Corps of Tasmania 
Here we report our findings on the
substantial potential of training as a strategic recruitment
and retention tool. It is
probable that most volunteers are motivated by both
self-interest and altruism, acting to reinforce each other.
This fits well with the evident desire for education:
training connects self-interested ‘learning new skills’
with altruistic urges to ‘assist the community’. The new
skills have little purpose or use in ambulance service
unless they are used to ‘assist the community’, and VAO
are unable to provide adequate emergency services to
the community without new skills. These two
motivations reinforce each other; our focus group data
strongly support this view as participants feared being
inadequate in emergency situations and desired training
to ensure competency.\\

94\% assisting the community
94\% learning new skills
 over 50\% reported various social benefits
 
 Our data reveal that retention is a problem in
Tasmania as most VAO remain for less than five years.
Emergency services understand that retention means
ensuring volunteers are happy with their role, and use
incentives, recognition and reimbursement as strategies
to ensure this. 
But training is not mentioned in this context.
However, we found training should be considered not
only as a recruitment tool, but also as a strong retention
tool. Training and related activities were by far the most
frequently stated activities enjoyed by respondents.\
 
 
 Increasing age correlated with less interest in
upgrading qualifications: 87.5\% of 18–30 year olds
wished to upgrade but only 46\% of those over 60 years
old had similar desires\\

Focus group findings revealed a strong consensus about
what is important to VAO. They felt a of lack adequate
support, particularly those located in remote regions.
Specifically, the VAO identified that one of their priority
needs was training, universally complaining of
inconsistent provision.\\

The irregular availability of the training modules
necessary to improve qualifications was another
identified problem.\\

poor training implementation made them lose confidence in
their ability to respond appropriately to emergencies.\\


\autocite{skoglund_not_2006}\\
high turnover rates are critical when there is a need for volunteers with specialized skills
or intensive training\\

The first 6 months of the volunteer experience are critical to their retention as the greatest loss occurs during this time\\

The Caring Hearts program leaking volunteers b/c (1) volunteers felt alone, (2) lack of ongoing training (only did a two-day initial training) (3) not enough opportunity to strut their identity (cultivate their role). Need processing time w/ other volunteers


\autocite{starnes_conceptual_2001}\\
Wymer and Starnes also note that one of the most frequent motivations for discontinuing volunteering is inadequate training\\

There is little research information available regarding turnover rates be-
cause few organizations have tracked such data. However, turnover rates can
be computed by applying the formula the U.S. Department of Labor uses for
the private sector
\begin{equation}
\ddfrac{Count \ of \ volunteer\ separations \ during \ the \ month}{Count\ of \ volunteers \ at \ midmonth} * 100
\end{equation}

Researchers and volunteer program managers have found that people’s mo-
tives for continuing volunteer service usually differ from their original mo-
tives to serve

Uncontrollable reasons include moving away, returning to
school or work, personal health problems, death, and low-income levels. Con-
trollable reasons include lack of transportation; other volunteer responsibili-
ties; work and family obligations; communication, status, and acceptance
problems between volunteers, paid staff, and clients; unrealistic expectations;
unclear roles; inadequate training; and insufficient use of the volunteer staff\\

The National Center for Nonprofit Boards (NCNB) recommends that
boards of directors include the organization’s volunteer programs in their
oversight efforts because volunteers save the organization money, serve as
ambassadors in the local community, and provide an additional source of feed-
back\\

It appears the highest turnover rates occur after three
months, six months, and twelve months of volunteering (Fischer and Schaffer
1993, Gidron 1978). Fischer and Schaffer suggest volunteers start their service
in a “honeymoon” stage of euphoria and self-congratulations but regress to a
“post-honeymoon blues” phase after gaining some experience. This regression
may occur when volunteers realize they are not able to accomplish what they
had hoped, that the organization doesn’t represent the values or issues they
originally thought, or they began service with too much idealism.\\

Recognition Programs. To encourage and motivate volunteers to continue
their service Volunteer program managers may:
a. prepare handwritten notes of thanks.
b. publicize accomplishments in newsletters and local newspapers.
c. offer volunteers opportunities to write and publish stories about their
volunteer experiences.
d. inform spouses, children, parents, close friends, and employers about
the volunteer’s contributions.
e. nominate outstanding volunteers for external awards.
f. distribute calendars displaying member’s pictures to the community.
g. establish a series of graduated awards that recognize higher/continuing
degrees of service.
h. give direct praise.
i. post positive comments from clients on public bulletin boards.
j. provide passes to community parks and recreation areas and coupons
from local businesses.
k. provide certificates, pins, and recognition dinners

\autocite{karl_give_2008}\\
Even though, as stated by Bernie DeKoven (Gilbet t 2001), nonprofit
organizations suffer from the 'serious syndrome,' we believe that
volunteers will be similar to other employees in their ratings of fun
activities. However, as is true with most leisure or pastime activities
(sports and hobbies), it is expected that there will be wide variation
atnong volunteers in the degree to which certain activities are considered
fun.\\

An
examination of the correlations shows that, as predicted, volunteers
who experienced higher levels of fun in their volunteer workplace also
had higher satisfaction with their volunteer positions and lower
turnover intentions. Age and gender were not related to attitudes
toward fun or experienced fun.\\

So, should we give volunteers something to smile about? Our
findings suggest that managers of volunteers could expect positive
outcomes from creating a fun work environment or implementing a
culture of fun. When asked about their attitudes toward workplace
fun. most volunteers in our sample reported that having fun in their
volunteer workplace was important to them and that it was
appropriate and led to positive consequences for the organization.\\

Behaving in what may be perceived as silly and
frivolous activities may be an insult to those who feel they are
volunteering for serious and honorable reasons. This notion is
somewhat supported by Morrow-Howell and Mui (1989) who found
that task-oriented volunteers (as opposed to people-oriented) viewed
social interaction as an unnecessary distraction while volunteering.\\

Similar to Karl and Harland (2005), we found significant age and
gender differences in what was considered fun and not fun by our
sample of volunteers.\\

However, given that we found
considerable variation in preferences for certain types oC fun
activities, managers of volunteers would need to exercise caution in
how they implement such efforts into the work environment.\\








\autocite{shields_young_2009}\\
Volunteer retention rates are also influenced by several factors perceived by the individual. Specifically, volunteers express high levels of connectedness (feeling part of a group to which one feels they share goals, values, respect, and trust), uniqueness (feeling that one has a unique combination of talents and personality to contribute to the organization), and power (feeling that one can make a difference)\\

young adults have been identified as an under-represented
age group in volunteering . The image of
volunteering may need to be altered to reflect positive and relevant images
to accommodate the needs of young adults. A large proportion of young
people perceive volunteering as not being socially or personally attractive
. Many young adults perceive volunteering as
being boring and involving older people who will not appreciate their skills.
Perceptions such as this need to be overcome as young people represent an
attractive source of volunteers for nonprofit organizations and, thus, are an
important focus for nonprofits’ recruitment efforts . Recognition should be given to the fact that the three most
common barriers to volunteering are lack of time, lack of interest, and ill
health . Young adults would be
particularly venerable to the first two barriers, thus representing a special
recruiting challenge. \\

Decisive research on why people volunteer discovered that
volunteers were motivated by both altruistic and egoistic motives. A widely quoted attempt to categorize volunteering
motivation resulted in six motives for volunteering. Two of the six motives
were related to career and personal enhancement, two were personal development (protective and understanding), 
and two dealt with relationships with others (social and values)\\

Volunteer retention rates
are also influenced by several factors perceived by the individual.
Specifically, volunteers express high levels of connectedness (feeling part
of a group to which one feels they share goals, values, respect, and trust),
uniqueness (feeling that one has a unique combination of talents and
personality to contribute to the organization), and power (feeling that one
can make a difference)\\

Peterson (2004) found that younger volunteers were motivated by
financial and career success and older volunteers were motivated by social
responsibilities and a greater . This study found age to
be the best criteria for selection of recruitment strategies. Australian
researchers, however, concluded that motivations for volunteering did not
differ by age, and that generic promotional and recruitment messages
would be equally effective for all age groups of volunteers\\

Callow (2004) recognized the advantages of
identifying different promotional appeals for targeting the retiree volunteer
segment. Instead of highlighting the similarities within the retiree segment,
Callow advocated future fragmenting the cluster into subgroups.\\

It has already been noted that individuals are often motivated to
volunteer because of egoistic reasons. Young adults are more likely to
volunteer to benefit their own self-interests and concern for their own
personal advancement\\

What promotional appeals for volunteering experiences will subgroups
of young adults find desirable? What are the most appropriate targeting
appeals for this segment? Do the motivations of young adult volunteers differ
from retiree volunteers significantly enough to warrant separate appeals?
How likely and when are young adults to volunteer? Can young adults with
particular desirable traits be recruited with the generic appeals?\\

An eight-item scale by Raskin and
Terry (1988) was used to measure the degree to which one views oneself as
a leader and desires to have influence over others\\

were more likely to volunteer out of egoism (less altruistic)
self-interest, a measure of concern for personal advancement included was
one of four components of vanity. Achievement concern was measured by
Netemeyer, Burton and Lichtenstein’s (1995) five-item scale. This variable
was included to capture a sense of personal enhancement and achievement.\\

Deeming positive attitudes towards working in a team and cooperative
behavior toward others as desirable traits for volunteers, Oliver and
Anderson’s (1994) five-item acceptance (teamwork/cooperation) scale was
included in the study.\\

Three components of mentoring, ability, and willingness (rapport
ability, support willingness, and relational willingness) were considered
admirable traits for potential volunteers and were measured by Pullins et al.’s
(1996) 15-item scale.\\

The promotional appeals that Callow identified for targeting potential retiree
volunteers proved to be effective for segmenting the young adult market, as
well. The four different appeals, based on high and low humanitarian and
social motivations, proved to be attractive to the students in the study\\

By appealing to and targeting subgroups with different motivations
individually, the maximum marketing efficiency can be reached within the
overall segment. Recruitment communications by nonprofit organizations
directed to young adults should highlight and emphasize the following types
of themes and positioning strategies, for instance:
When you volunteer at XYZ nonprofit organization...
...you will actively participate in important activities that make a
difference.
...you will have the opportunity to better yourself personally and
professionally.
...you will make a difference in the lives of others.
...you will make connections with others.\\

The volunteer market, in general, and regardless of life-stage, appears
to be motivated by a combination of humanitarianism and social factors.
Thus appeals to the young adult segment should emphasize these distinct
themes in a format that addresses the specific interests of the cohort groups
such as self-advancement and personal development, since efforts should be
made to accommodate the preferences and imperatives of young people to
be most effective\\

Given that almost a third
of the segment indicated that they are very likely to volunteer in the future, a
carefully executed marketing strategy can be instrumental in attracting and
recruiting the energetic and talented young adult segment who currently
believe they will volunteer in the future. With targeted appeals, the
opportunity to increase volunteerism and capture the energies of the young
adult volunteer segment exists.\\

Considering that professionally employed, college-educated individuals tend
to volunteer in higher numbers than other segments (Wymer, 2003), the fact
that the majority of the respondents had volunteered during their college
experience and felt the need to list volunteering on their resume is not
surprising. Motivation for volunteering is often employment-related.\\

It
would seem young adults can best be recruited by nonprofit organizations if
initial approaches are made, and relationships developed, while the segment
is still in college. The identified humanitarianism/social motivations appear
to apply to multiple life-stages. These appeals combined with the motivation
to start, and establish, a career should prove effective for recruitment efforts.




Generally, there are four components to any successful outreach campaign. \textit{Awareness building} announces the existence of volunteer opportunities with an organization, often using traditional approaches such as announcements in electronic media, promotional events, media tours, and press conferences. \textit{Curiosity satisfaction} provides targeted information to interested and curious individuals with the intention of moving them to become actual volunteers. Traditional approaches consist of strategically aired radio interviews and community-specific presentations at public meetings, neighborhood association meetings, and church group gatherings. \textit{Commitment fostering} means that an organization makes the volunteer sign up process as simple and direct as possible. Finally, \textit{data-driven feedback} means that well-defined objectives plus data collection and evaluation instruments are employed throughout the outreach campaign to create a cycle of adaptive management. 






% ----------------------------------------------------------------------------------------
% 	BIBLIOGRAPHY
% ----------------------------------------------------------------------------------------
          

\printbibliography



\end{document}
